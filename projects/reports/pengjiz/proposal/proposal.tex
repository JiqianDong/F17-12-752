\documentclass[11pt, letterpaper, headings=standardclasses]{scrartcl}

\usepackage{hyperref}
\usepackage{tikz}
\usetikzlibrary{automata, positioning}

\begin{document}
\title{Non-Intrusive Load Monitoring Using Prior Models of
  Appliance Usage}
\subtitle{12752 Project Proposal}
\author{Pengji Zhang}
\date{Nov 24, 2017}

\maketitle

\section{Introduction}

Non-intrusive load monitoring (NILM) is aimed to estimate the
electricity consumption of each appliance from the aggregate
consumption, without installing any additional
sensors.~\cite{hart1992nonintrusive} Because of the rapid growth in
smart meter deployments all over the world---which provides an ideal
platform for collecting electricity consumption data and communicating
with consumers~\cite{kim2011unsupervised, parson2012non}---we can
expect NILM will be of great importance in the future for the
management of the power system.~\cite{kim2011unsupervised}

Many approaches have been proposed to solve the NILM problem. Even
though the methods of those solutions are very different from each
other, most of them relies heavily on the power load data to train the
model. For example, Parson et al.\@ used the aggregated power load
data to learn the transition matrix and the emission matrix for each
appliance in the setting of a hidden markov
model;~\cite{parson2012non} Hart et al.\@ used a model to detect and
attribute the significant changes in the power
load.~\cite{hart1992nonintrusive} However, given the rise of
\emph{smart cities}, we could have a lot more information that can
help us to build a better model. For example, we may use the income
information of a person to predict the number of appliances and the
types of those appliances he/she has and tweak our model with that; we
may use the traffic flow data to predict the time that a person will
return back home and start using appliances. Thus, for this project,
we would like to explore the feasibility of building a NILM solution
with \emph{not only the power load and the data for the power system,
  but also the ambient information and the information of the
  residents.}

\section{Data}

The data set we are going to use is the Dutch Residential Energy
Dataset (DRED).~\cite{uttama2015loced} The data set contains the
appliance level consumption data, the ambient information (indoor and
outdoor temperatures, wind speed, humidity and precipitation) and the
room-level occupancy information of one household. We will split the
data set into a training set and a test set because we could not find
a similar data set from another household.

\section{Methods}

We will use the hidden markov model to decode the states of
appliances. First, for each appliance we will build models for
estimating the transition probabilities from one state to another
state and the emission probabilities of the power consumption of one
state, with the temperature, occupancy and time features. We plan to
use a softmax regression model for this. Then, we will apply iterative
Viterbi algorithm to find the most likely state chain for each
appliance, which is the same as the method proposed by Parson et
al.~\cite{parson2012non}

\section{Expected Results}

We expect to have a method for NILM based on HMM but with information
from various sources. With this method, the HMM model can be
\emph{built} efficiently and accurately for each household based on
the data other than the power load.

\nocite{*}
\bibliographystyle{unsrt}
\bibliography{references.bib}

\end{document}

%%% Local Variables:
%%% mode: latex
%%% TeX-master: t
%%% End:
